This paper studies the circular relationship between sovereign credit risk, government fiscal and debt policy, and output. I consider a sovereign default model with fiscal policy and private capital accumulation. I show that, when fiscal policy responds to borrowing conditions in the sovereign debt market, multiple equilibria exist where the expectations of lenders are self-fulfilling. In the bad equilibrium, pessimistic beliefs make sovereign debt costly. The government substitutes borrowing with taxation, which depresses private investment and future output, increases default probabilities and verifies lenders' beliefs. This result is reminiscent of the European debt crisis of 2010-12: while recessionary, fiscal austerity may be the government best response to excessive borrowing costs during a confidence crisis. 