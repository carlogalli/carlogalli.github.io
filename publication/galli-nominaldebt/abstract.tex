This paper explores the trade-off between strategic inflation and default for a set of large emerging market economies that borrow mostly in their local currency. Using over-the-counter derivatives data, I find a robust, positive correlation between default risk, inflation risk, and realised inflation. I use these facts to discipline a quantitative sovereign default model where a government issues debt in domestic currency and lacks commitment to both fiscal and monetary policy. I show that simple models of debt dilution via default and inflation have counterfactual implications, as default and inflation are substitutes and co-move negatively. I highlight the role that monetary financing plays to match the data, allowing inflation to serve a second purpose: in bad times, seignorage is especially useful as a flexible source of funding when other margins may be hard to adjust. The model matches the positive correlation between inflation and default risk, and allows to quantitatively evaluate its implication for the trade-off between the insurance benefits of nominal debt and the ex-ante cost of a further source of time inconsistency.   